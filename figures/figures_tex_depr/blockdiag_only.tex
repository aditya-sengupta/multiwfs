\documentclass{article}
\usepackage[margin=0.5in]{geometry}
\usepackage{custom}
\usepackage{graphicx,tikz}
\usepackage{xcolor}
\usetikzlibrary{shapes,arrows,positioning,arrows.meta}

\makeatletter
% the contents of \squarecorner were mostly stolen from pgfmoduleshapes.code.tex
\def\squarecorner#1{
    % Calculate x
    %
    % First, is width < minimum width?
    \pgf@x=\the\wd\pgfnodeparttextbox%
    \pgfmathsetlength\pgf@xc{\pgfkeysvalueof{/pgf/inner xsep}}%
    \advance\pgf@x by 2\pgf@xc%
    \pgfmathsetlength\pgf@xb{\pgfkeysvalueof{/pgf/minimum width}}%
    \ifdim\pgf@x<\pgf@xb%
        % yes, too small. Enlarge...
        \pgf@x=\pgf@xb%
    \fi%
    % Calculate y
    %
    % First, is height+depth < minimum height?
    \pgf@y=\ht\pgfnodeparttextbox%
    \advance\pgf@y by\dp\pgfnodeparttextbox%
    \pgfmathsetlength\pgf@yc{\pgfkeysvalueof{/pgf/inner ysep}}%
    \advance\pgf@y by 2\pgf@yc%
    \pgfmathsetlength\pgf@yb{\pgfkeysvalueof{/pgf/minimum height}}%
    \ifdim\pgf@y<\pgf@yb%
        % yes, too small. Enlarge...
        \pgf@y=\pgf@yb%
    \fi%
    %
    % this \ifdim is the actual part that makes the node dimensions square.
    \ifdim\pgf@x<\pgf@y%
        \pgf@x=\pgf@y%
    \else
        \pgf@y=\pgf@x%
    \fi
    %
    % Now, calculate right border: .5\wd\pgfnodeparttextbox + .5 \pgf@x + #1outer sep
    \pgf@x=#1.5\pgf@x%
    \advance\pgf@x by.5\wd\pgfnodeparttextbox%
    \pgfmathsetlength\pgf@xa{\pgfkeysvalueof{/pgf/outer xsep}}%
    \advance\pgf@x by#1\pgf@xa%
    % Now, calculate upper border: .5\ht-.5\dp + .5 \pgf@y + #1outer sep
    \pgf@y=#1.5\pgf@y%
    \advance\pgf@y by-.5\dp\pgfnodeparttextbox%
    \advance\pgf@y by.5\ht\pgfnodeparttextbox%
    \pgfmathsetlength\pgf@ya{\pgfkeysvalueof{/pgf/outer ysep}}%
    \advance\pgf@y by#1\pgf@ya%
}
\makeatother

\pgfdeclareshape{square}{
    \savedanchor\northeast{\squarecorner{}}
    \savedanchor\southwest{\squarecorner{-}}

    \foreach \x in {east,west} \foreach \y in {north,mid,base,south} {
        \inheritanchor[from=rectangle]{\y\space\x}
    }
    \foreach \x in {east,west,north,mid,base,south,center,text} {
        \inheritanchor[from=rectangle]{\x}
    }
    \inheritanchorborder[from=rectangle]
    \inheritbackgroundpath[from=rectangle]
}

\begin{document}
\tikzstyle{block} = [draw, rectangle, minimum height=1cm, minimum width=1.4cm]
\tikzstyle{sum} = [draw, fill=white, circle, node distance=1cm, text width=0.5cm]
\tikzstyle{input} = [coordinate]
\tikzstyle{output} = [coordinate]
\tikzstyle{pinstyle} = [pin edge={to-,thin,black}]
\tikzstyle{gain} = [draw, fill=white, isosceles triangle, isosceles triangle apex angle = 60, shape border rotate=#1]

\begin{figure}
    \centering
    \begin{tikzpicture}[auto,>=latex', align=center, node distance=2cm]
    % Blocks for the plant.
    \node [name=disturbance, red!80!blue] {$\phi$};
    \node [sum, name=sum, right of=disturbance] {+};
    \node [name=error, blue!80!red, above right of=sum] {$\varepsilon$};
    \node [input,name=wfssplit, right of=disturbance, node distance=3cm] {};
    \node [sum, name=fastncpsum, right of=wfssplit, node distance=1cm, yshift=1.4cm] {$+$};
    \node [sum, name=slowncpsum, right of=wfssplit, node distance=1cm, yshift=-1.4cm] {$+$};
    \node [name=fastncp, above of=fastncpsum, red!80!blue, node distance=1cm] {$L_\text{fast}$};
    \node [name=slowncp, above of=slowncpsum, red!80!blue, node distance=1cm] {$L_\text{slow}$};
    \node [block,draw, name=fastwfsdelay, right of=fastncpsum] {$\frac{1 - e^{-sT}}{sT}$}; 
    \node [block,draw, name=slowwfsdelay, right of=slowncpsum] {$\frac{1 - e^{-sRT}}{sRT}$};
    \node [sum, name=fastnoisesum, right of=fastwfsdelay, node distance=2cm] {$+$};
    \node [sum, name=slownoisesum, right of=slowwfsdelay, node distance=2cm] {$+$};
    \node [red!80!blue] (fastnoise) at (fastnoisesum |- fastncp.base) {$N_\text{fast}$};
    \node [red!80!blue] (slownoise) at (slownoisesum |- slowncp.base) {$N_\text{slow}$};
    \node [name=sfast, right of=fastnoisesum, blue!80!red] {$s_\text{fast}$};
    \node [name=sslow, right of=slownoisesum, blue!80!red] {$s_\text{slow}$};

    % Blocks for the controller and feedback arm.
    \node [block, draw, below of=slownoisesum, name=slowcontroller, xshift=0.75cm, node distance=3cm] {$C_\text{slow}$};
    \node [block, draw, below of=slowcontroller, name=fastcontroller] {$C_\text{fast}$};
    \node [block, draw, left of=slowcontroller, name=slowcompute, node distance=2.75cm] {$e^{-sT_{c,\text{slow}}}$};
    \node [block, draw, left of=fastcontroller, name=fastcompute, node distance=2.75cm] {$e^{-sT_{c,\text{fast}}}$};
    \node [sum, left of=slowcompute, name=controlsum, node distance=3cm, yshift=-1cm] {$+$};
    \node [block, below of=sum, name=zoh] {$\frac{1 - e^{-sT}}{sT}$};

    % All the direct arrows.
    \draw[->] (disturbance) -- (sum);
    \draw[->] (fastncpsum) -- (fastwfsdelay);
    \draw[->] (fastncp) -- (fastncpsum);
    \draw[->] (fastwfsdelay) -- (fastnoisesum);
    \draw[->] (fastnoisesum) -- (sfast);
    \draw[->] (fastnoise) -- (fastnoisesum);
    \draw[->] (slowncpsum) -- (slowwfsdelay);
    \draw[->] (slowncp) -- (slowncpsum);
    \draw[->] (slowwfsdelay) -- (slownoisesum);
    \draw[->] (slownoisesum) -- (sslow);
    \draw[->] (slownoise) -- (slownoisesum);
    \draw[->] (fastcontroller) -- (fastcompute);
    \draw[->] (slowcontroller) -- (slowcompute);
    \draw[->] (zoh) -- (sum);

    % Anchor nodes and corresponding paths for indirect arrows.
    \draw [->] (sum) -| (error);
    \draw [-] (sum) -- (wfssplit);
    \draw [->] (wfssplit) |- (slowncpsum);
    \draw [->] (wfssplit) |- (fastncpsum);
    \draw [->] (fastcompute) -| (controlsum);
    \draw [->] (slowcompute) -| (controlsum);
    \draw [->] (controlsum) -| (zoh);

    \node [input, left of=sfast, node distance=1cm, name=sfastsplit] {};
    \node [input, below of=sfastsplit, name=sfastturnaway] {};
    \node [input, right of=sfastturnaway, name=sfastturndown] {};
    \draw [->] (sfastsplit) -- (sfastturnaway) -- (sfastturndown) |- (fastcontroller);

    \node [input, left of=sslow, node distance=1cm, name=sslowsplit] {};
    \node [input, below of=sslowsplit, name=sslowturnaway, node distance=1cm] {};
    \node [input, right of=sslowturnaway, name=sslowturndown, node distance=1cm] {};
    \draw [->] (sslowsplit) -- (sslowturnaway) -- (sslowturndown) |- (slowcontroller);

    % Explanatory text around blocks.
    \node [name=X, right of=slowncpsum, node distance=1cm] {};
    \node [name=Xlabel, above of=X, node distance=0.25cm, blue!80!red] {$X$};
    \node [name=Y, right of=fastncpsum, node distance=1cm] {};
    \node [name=Ylabel, above of=Y, node distance=0.25cm, blue!80!red] {$Y$};
    \filldraw[ blue!80!red] (X) circle (1pt);
    \filldraw[ blue!80!red] (Y) circle (1pt);

    \node [name=slowwfstext, above of=slowwfsdelay, black!40!green, node distance=0.9cm] {\small Slow WFS delay};
    \node [name=fastwfstext, above of=fastwfsdelay, black!40!green, node distance=0.9cm] {\small Fast WFS delay};
    \node [name=slowcomputetext, above of=slowcompute, black!40!green, node distance=0.9cm] {\small Slow computation};
    \node [name=fastcomputetext, above of=fastcompute, black!40!green, node distance=0.9cm] {\small Fast computation};
    \node [name=zohtext, below right of=zoh, black!40!green, node distance=1.3cm] {\small DM zero-\\order hold};
\end{tikzpicture}

    \caption{block diagram. I'm going to add a full line of useless text so we can see if the diagram's centered relative to it.}
    \label{fig:blockdiagram}
\end{figure}
    
\end{document}